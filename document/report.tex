% !TeX program  = XeLaTeX
% !TeX encoding = UTF-8

\documentclass[UTF8]{ctexart}

\usepackage[
    colorlinks=true,
    linkcolor=black,
    urlcolor=black,
    bookmarksopen=true
    pdfauthor={蒋翌坤},
    ]{hyperref}
\usepackage{bookmark}
\usepackage{graphicx}
\usepackage[export]{adjustbox}
\usepackage{amsmath}
\usepackage{pgfplotstable}
\usepackage{booktabs}
\usepackage{tabularx}
\usepackage{indentfirst}
\usepackage[inline]{enumitem}
%\usepackage{chngcntr}
%\counterwithin{figure}{section}
%\counterwithin{table}{section}

\usepackage{template}

\newcommand{\hmwkTitle}{疫情下的谣言传播可视分析系统}
\newcommand{\hmwkClass}{程序设计与数据可视化}


\begin{document}

\clearpage
\makecover

\pagebreak

\tableofcontents

\pagebreak

\section{数据预处理}

本次研究的数据集为~\verb|rumor.csv|,其中包含了366条谣言数据。鉴于数据背景、来源、变量含义以及预期目标已经在作业中给出,此处不再赘述。
在数据处理、分析前,我们将数据存储在~\verb|pandas.DataFrame|~类中,以便于后续的数据处理与可视化分析。

在此节中,我们将对数据集进行预处理,为接下来的数据可视化分析打下基础。
首先,选取数据集的前5条谣言数据,观察数据集的部分内容及描述性统计,确定预处理数据的步骤。
数据集前5条谣言数据如表~\ref{tab:数据集内容}~所示。
数据集各变量在~\verb|DataFrame|~中的类型及部分描述性统计如表~\ref{tab:数据集变量类型、部分描述性}~所示。

\begin{table}[!ht] \centering
    \begin{tabularx}{\linewidth}{X c c c c c c c}
        \toprule
        date & source & content & province & user\_0 & \dots & user\_17 & like \\
        \midrule
        2022-02-18 & 北京日报客户端 & 有人从香港\dots & 广东 & 0 & \dots & 0 & 0 \\
        2022-02-18 & 大河报、羊城晚报 & 近日,社交\dots & 河南 & 0 & \dots & 0 & 41 \\
        2022-02-17 & 南方都市报、大众网 & 近日,网传\dots & 湖南 & 0 & \dots & 0 & 26 \\
        2022-02-17 & 江苏省互联网举报中心 & 近日,网传\dots & 江苏 & 0 & \dots & 0 & 31 \\
        2022-02-16 & 中国新闻网 & 在塞企业机\dots & \color{red}{NaN} & 0 & \dots & 0 & 0 \\
        \bottomrule
    \end{tabularx}
    \caption{数据集前5条谣言数据}
    \label{tab:数据集内容}
\end{table}

\begin{table}[!ht] \centering
    \begin{tabularx}{\linewidth}{X c c c c c c c c}
        \toprule
        变量名 & date & source & content & province & user\_0 & \dots & user\_17 & like \\
        \midrule
        类型 & object & object & object & object & int64 & \dots & int64 & int64 \\
        非空值 & 366 & 360 & 366 & 277 & 366 & \dots & 366 & 366 \\
        均值 & - & - & - & - & 0.19 & \dots & 0.02 & 72.59 \\
        最小值 & - & - & - & - & 0 & \dots & 0 & 0 \\
        最大值 & - & - & - & - & 1 & \dots & 1 & 2495 \\
        \bottomrule
    \end{tabularx}
    \caption{数据集变量类型及部分描述性统计}
    \label{tab:数据集变量类型、部分描述性}
\end{table}

分析表~\ref{tab:数据集内容}~和表~\ref{tab:数据集变量类型、部分描述性}~内容,可以发现以下几个问题
\footnote{注意到数据集中user\_0到user\_17变量应该为Bool类型变量,但目前是int64类型的0-1变量,不妨碍分析。}:
% NOTE: 这一点不是很确定要不要写
\begin{enumerate}
    \item 数据集中source、province变量存在NaN值,需要对这些缺失值进行处理;
    \item 数据集中date变量应该为Datetime类型,但目前是object类型,需要进行类型转换才能进行与时间相关的分析;
    \item 数据集中like变量最小值、最大值差距很大,而均值很小,说明like变量的分布严重左偏。
\end{enumerate}

根据上述通过观察得到的问题,我们的预处理分为以下几个步骤\footnote{由于数据集中的每条谣言数据主要由谣言文本构成,因此不涉及离群值处理}:
\begin{enumerate}
    \item 数据缺失值处理。对于source、province变量的缺失值,由于谣言来源或涉及省份和谣言内容紧密相关,可以通过观察谣言内容结合搜索引擎,手动将这些缺失值填上;
    对于那些无法分辨出谣言来源或涉及省份时,填充数据或删除该条谣言数据都不合理,因此保留NaN值,
    在后续涉及source或province的分析过程中,忽略这些缺失值对应的谣言数据。
    \item 数据重复值处理。由于重复值的出现可能会影响分析结果,而数据集中谣言数量较小,可以通过观察法来判断数据集中是否存在重复值。
    % 如果存在重复值,则以谣言发布时间最早的为准。
    % NOTE: 如果有重复值,要写处理的方法
    \item 数据类型转换。将数据集中的date变量转换为Datetime类型。 % NOTE: user\_0到user\_17变量可能会改
    \item 变量数值变换。将数据集中的like变量进行log变换,使其分布更加接近正态。 % NOTE: 可能会变为box-cox变换
    \item 分词处理。将数据集中的source、content变量通过~\verb|jieba|~库进行分词处理。
    在分词时,利用自定义停用词表\footnote{见附录~\ref{sec:停用词表}~}中的词作为停用词,不进行分词。
    分词后的结果(一个list)保存在新的变量source\_token、content\_token中。 % NOTE: 可能source\_token没必要
\end{enumerate}

% TODO: 数据预处理后的结果


\section{数据可视化分析}

\subsection{描述性分析}

% 不同日期的谣言数量

\subsection{地理空间分析}


\subsection{时间序列分析}


\subsection{文本分析}

% 词云;TFIDF

\subsection{综合分析}

\pagebreak

\section{总结与展望}


\appendix
\section{附录}

\subsection{停用词表} \label{sec:停用词表}

\end{document}